\documentclass[letterpaper,11pt]{article}

\usepackage[margin=1in]{geometry}
\usepackage{times}
\usepackage[compact]{titlesec}
\usepackage{amsmath,amssymb,amsthm}
\usepackage{graphicx}
\usepackage{subcaption}
\usepackage{hyperref}
\usepackage[T1]{fontenc}
\usepackage[giveninits=true,style=mla]{biblatex}
%\usepackage[style=mla]{biblatex}
\usepackage{tabularx}
\usepackage{cleveref}
\usepackage{tikz}

\usetikzlibrary{arrows.meta}
\usetikzlibrary{shapes.geometric}
\usetikzlibrary{shapes.multipart}
\usetikzlibrary{cd}

\addbibresource{publication24.bib}

\graphicspath{
    {Figs/},{figures/}
}

\titleformat*{\paragraph}{\itshape}

\newcommand{\cF}{{\mathcal{F}}}
\newcommand{\cX}{{\mathcal{X}}}

\newcommand{\dg}[1]{{\textcolor{blue}{#1}}}

\begin{document}

\thispagestyle{empty}
\noindent\textbf{ANNUAL REPORT}\\[1cm]
\centerline{\textbf{\Large Unified Large-Scale Theoretical and Computational Frameworks}}
\centerline{\textbf{\Large for Invariance and Composition of Open Hybrid Dynamical Systems}}\\[1cm]

\renewcommand\arraystretch{1.5}
\begin{tabularx}{1.0\textwidth}{>{\bfseries}lX}
Award Number & FA9550-23-1-0400\\
Report Type & Annual\\
Reporting Period & July 2, 2024 - July 1, 2025\\
Distribution Statement & Distribution A - Approved for public release\\
Program Officer Name & Dr. Frederick Leve\\
Principal Investigator Name & Dr. Taeyoung Lee\\
Project Title & Unified Large-Scale Theoretical and Computational Frameworks for Invariance and Composition of Open Hybrid Dynamical Systems\\
%
Abstract & TBA
%This project is to establish both theoretical and computational frameworks to analyze and certify the intriguing behaviors of complicated open hybrid dynamical systems. 
%In particular, the objective is to identify and construct the inherent structures of hybrid dynamics, such as topological properties and invariances, that can be preserved under the interaction with uncertain environments and composition over a complex network. 
%During YR1, we investigated 15 topics in the area of geometry, topology, openness, scalability, and category of hybrid systems. 
\end{tabularx}

\clearpage\newpage

\tableofcontents

\clearpage\newpage
\setcounter{page}{1}
\section{Introduction}
This project will develop theoretical and computational frameworks to analyze and certify the behavior of complex open hybrid dynamical systems, such as embodied artificial intelligence,  quantum systems, and biological systems. 
The objective is to identify and construct the inherent structures of hybrid dynamics, such as topological properties and invariances, that can be preserved under the interaction with an uncertain environment and composition over a complex network.
The novelty lies in establishing a trustworthy computational foundation that is carefully constructed in conjunction with the underlying geometry, leading to a significant generalization capacity and computational efficiency in understanding the global topological properties of complex, composable hybrid systems. 

This will be achieved by multidisciplinary collaborative efforts in dynamical systems and theoretical computer science. 
In particular, we focus on three research thrusts: discovery of geometric and topological structures; investigation of the uncovered structures in open, uncertain environments and composition; extension to gene regulatory networks/multi-agent systems and further generalization to scalable composition and certification, as illustrated in \Cref{fig:overview}.

\usetikzlibrary{matrix, positioning, fit, backgrounds}
\begin{figure}[h]
    \scriptsize
    \centerline{
        \begin{tikzpicture}[
            mymatrix/.style={matrix of nodes, nodes=typetag, row sep=7pt, align = center},
            typetag/.style={draw, thick, fill=white, inner sep=1.5ex, anchor=west, text width = 0.27\textwidth, align = center},
            mycontainer/.style={draw=gray, inner sep=1ex},
            title/.style={draw=none, fill=none, inner sep=0pt, align=center, font=\bfseries},
            analytic/.style={draw},
            mixed/.style={draw, dash dot},
            comp/.style={draw, dotted}
            ]
            \matrix[mymatrix] (GT) {
                |[title]| (i) Geometry \& Topology\\
                |[analytic]| {Geometry and Topology\\ (Bloch, Clark)} \\
                |[mixed]| {Homological Dynamics\\ (Mischaikow, Kalies)} \\
                |[comp]| { Topological Dynamics Learning\\ (Ghaffari, Bloch, Clark) }\\ 
            };
            \matrix[mymatrix, right=20pt of GT.north east, matrix anchor=north west] (OS) {
                |[title]| (ii) Openness \& Scalability\\
                |[analytic]| { Geometry and Topology\\ (Bloch, Clark)} \\
                |[mixed]| { Stochastic Hybrid Network\\ (Lee, Mischaikow, Kalies)} \\
                |[analytic]| { Unified Composition Framework \\ (Guralnik, Vasudevan, Ghaffari) }\\ 
            };
            \matrix[mymatrix, right=20pt of OS.north east, matrix anchor=north west, row sep=5pt] (EV) {
                |[title]| (iii) Extension \& Verification\\
                |[mixed, inner sep=3pt]| { Gene Regulatory Networks\\[-0.05cm] (Mischaikow, Kalies)} \\
                |[mixed, inner sep=3pt]| { Cooperative Multi-agent Systems\\[-0.05cm] (Lee, Guralnik)} \\
                |[comp, inner sep=3pt]|{ Scalable Composition\\[-0.05cm] (Vasudevan, Ghaffari, Guralnik)} \\
                |[comp, inner sep=3pt]| { Formal Verification\\[-0.05cm] (Vasudevan, Lee, Mischaikow) }\\ 
            };
            \node[mycontainer, fit=(GT)] (GTBox) {};%, label={[text width=0.27\textwidth] below:discover topological properties with three distinct approraches}] (GTBox) {};
            \node[mycontainer, fit=(OS)] (OSBox) {};
            \node[mycontainer, fit=(EV)] (EVBox){};

            \begin{scope}[on background layer]
                \node[mycontainer, fit=(GTBox), fill=gray!10, inner sep = 0pt] {};
                \node[mycontainer, fit=(OSBox), fill=gray!10, inner sep = 0pt] {};
                \node[mycontainer, fit=(EVBox), fill=gray!10, inner sep = 0pt] {};
            \end{scope}

            \draw[arrows={-Triangle[angle=30:6pt]}, thick] (GTBox) -- (OSBox);
            \draw[arrows={-Triangle[angle=30:6pt]}, thick] (OSBox) -- (EVBox);
            \draw[arrows={-Triangle[angle=30:6pt]}, thick] (EVBox.south) -- ++(0,-8pt) -- ([yshift=-8pt]GTBox.south) -- (GTBox.south);
        \end{tikzpicture}
    }
    \DeclareRobustCommand\captionsolid{\tikz[baseline=-0.6ex]\draw[thick] (0,0)--(0.5,0);}
    \DeclareRobustCommand\captiondashdot{\tikz[baseline=-0.6ex]\draw[thick, dash dot] (0,0)--(0.5,0);}
    \DeclareRobustCommand\captiondotted{\tikz[baseline=-0.6ex]\draw[thick, dotted] (0,0)--(0.5,0);}
    \caption{Proposed research: (i) we will discover geometric and topological properties of hybrid systems with three distinct approaches: theoretical analysis (\captionsolid), mixed theoretical analysis and computation (\captiondashdot), and computation (\captiondotted); (ii) next, we will investigate how such topological properties are affected by environments, uncertainties, and large-scale compositions; (iii) these will be further extended to gene regulatory networks and multi-agent systems, and they will be verified with formal methods.}\label{fig:overview}
\end{figure}

\newpage

\section{Geometry and Topology}
\paragraph*{Outcome}

\subsection{Linear affine hybrid system (Clark)}\label{affine HS}
Linear systems is a common introduction to the theory of ordinary differential equations/dynamics and control theory. In \cite{clark_linear}, a notion of linear hybrid systems is studied. Although temporally-triggered linear hybrid systems have garnered much attention, spatially-triggered systems have not - in part because the solution map for these latter systems are no longer linear (in the initial conditions). Moreover, these systems possess complicated topological properties as beating/blocking always occurs. However, these systems are too simple to support Zeno solutions as a bias is required to prevent the origin from being a fixed-point of the dynamics. As such, the simplest class of hybrid systems that are sufficiently complicated to exhibit the hallmark pathologies of hybrid systems are \textit{affine hybrid systems}. In \cite{clark_linear}, the optimal control problem is considered for these affine hybrid systems and the existence/uniqueness of the co-adjoint equation is studied in the context of exceptional solutions (beating/blocking/Zeno). We are developing numerical schemes to efficiently handle these problems.

\subsection{Hybrid Conley index theory (Kalies, Mischaikow)}\label{hybrid Conley}

This work introduces a combinatorial-topological approach for the  analysis of hybrid dynamical systems, \cite{KaRi-hybridCIT}. Given the inherent challenges in characterizing their long-term behavior and attractor structures due to discontinuities, the research addresses the fundamental problem of extracting robust topological and algebraic invariants via Conley index theory.

\paragraph*{Outcome} A typical approach in the setting of dynamical systems is to construct a combinatorial model and associate its dynamical structure with an open class of dynamical systems. In the hybrid setting, standard approaches to such discretizations are not directly applicable. We show that under certain conditions this challenge can be addressed by factoring through an associated semiflow whose underlying dynamics corresponds to the hybrid system. A computational pipeline can be implemented that leverages existing reconstruction theorems and software packages to characterize the global dynamics of the original hybrid system. We showcase that this approach is extendable to situations in which a hybrid system does not satisfy the trapping guard condition with the trade-off that the usual dynamical interpretations  of the resulting algebraic invariants is no longer directly applicable. 

\paragraph*{Future Directions} We are currently extending this approach to broader classes of hybrid systems. In connection with the efforts in Sections~2.3, 2.4, and~4.5, the development of mathematical theory is needed to characterize the dynamical structure of more general classes of hybrid systems using combinatorial and algebraic approaches, including
an understanding of how to intepret or reconstruct dynamics from the resulting algebraic invariants.

\subsection{Data-driven learning for hybrid systems (Mischaikow, Clark, Kalies)}
\paragraph*{Outcome}
In \cite{gameiro2025rigorouslycharacterizingdynamicsmachine} we argue that the classical theory of nonlinear dynamics based on the existence and structure of invariant systems is too rich to be learned from data.
Furthermore, we prove that dynamics based on order theory and algebraic topology (Conley theory) can be learned given sufficient data and sufficient computational resources.
In \cite{gameiro2025datadrivenidentificationattractorsusing} we explore the challenges of using machine learning to identify the structure of attracting regions using data sets generated by ODEs.

We have begun the process of extending these ideas to the setting of hybrid systems. 
In particular, in joint work with R. Kuske (GaTech) we have been developing computational tools to characterize the global dynamics of a vibro-impact energy harvester (see Serdukova, Kuske, and Yurchenko, “Post-grazing dynamics of a vibro-impacting energy generator” (2021)).
Our code is now capable of characterizing the lattice of attractors even though the  phase space for the data provided by R. Kuske lies on a manifold as opposed to Euclidean space.

More specifically, using data coming from simulations provided by R. Kuske we are attempting to characterize lattices of attractors and compute Conley indices. 

\paragraph*{Future Directions}
A fundamental challenge is to develop a computational Conley index theory that is capable of characterizing dynamics even though hybrid systems exhibit discontinuous dynamics.
As is argued in Section~\ref{sec:homologicaldynamicsHS} we have achieved this in the context of DSGRN where the goal is to characterize the dynamics of a ODE.
However, in the data-driven setting it is more natural to assume that we are trying to characterize the dynamics induced by a map.
In this context it appears that discontinuities range from ignorable to local to global. 
As of yet these terms are not defined, but the intuition is as follows.
We are using a cellular theory of homological computations, thus discontinuities that fall within individual cells have no impact on the homological computations and can be easily regularized.
Local discontinuities lead to non-acyclic images -- on a technical level this violates the acyclic carrier theorem that is often used to prove that maps induced by homology are well-defined.
Earlier work by Mischaikow and co-authors suggests possibilities for addressing this setting.
Global discontinuities are a wide open challenge.
In summary we need to define these terms, develop algorithms for identifying these settings, and develop mathematical theory that leads to proper regularizations. 



\subsection{Homological dynamics of hybrid system (Mischaikow, Guralnik, Kalies)}
\label{sec:homologicaldynamicsHS}
With respect to hybrid systems is it useful to view
Dynamical Systems Generated by Regulatory Networks (DSGRN) as a mathematical framework that allows one to begin with a system of ODES with discontinuous vector fields and through explicit combinatorial and algebraic topological identify continuous dynamics that is consistent with  appropriate  regularizations of the ODEs.
We emphasize that the appropriate regularizations can be implicit as opposed to explicit -- with DSGRN we succeed in providing rigorous descriptions of dynamics without explicit expressions of the regularized system.
The mathematical theory this perspective is made clear in \cite{gameiro2024globaldynamicsordinarydifferential} and further experimental validation is provided in \cite{kepley2024globalanalysisregulatorynetwork}.

\paragraph*{Future Directions}
Our perspective is that it is the multiscale nature of hybrid systems that leads to the perspective that these are not continuous systems, while the real underlying physical system behaves continuously.
Our goal is to extend the above mentioned DSGRN philosophy and techniques to hybrid systems. 
In particular, using order theory and homological tools we will provide a rigorous global characterization of the dynamics without requiring specific modeling of the physics that occurs at the discontinuities.
The fundamental challenge is that interpretation of algebraic topological invariants typically assumes continuity.
We overcome this in DSGRN via a two step process: (i) providing a discrete representation of phase space that is rich enough to resolve the discontinuity, and (ii) providing a combinatorial model of dynamics that is compatible with the appropriate continuous model of dynamics.
Depending on the hybrid  systems of interest it may be necessary to modify both (i) and (ii).
%

\paragraph*{}
In a recent development~\cite{ClGu-weakened_topologies}, we have identified a procedure whereby a jump relation (AKA hybrid structure) on a state space gives rise to a weaker $T_0$ topology, with respect to which hybrid paths become continuous functions of time.
%
Moreover, the class of discontinuities allowed for a path in the state space, while it remains continuous with respect to the weakened topology, is broadened significantly, accommodating, for example, paths accumulating on a guard but not intersecting it transversely.
%
We intend to apply the notion of the weakened topology as the foundation of a formal framework for discussing hybrid dynamics while circumventing the need for state-identifications along resets in the description of hybrid state space, and hybrid time domains in the description of trajectories.
%
Similar to the underlying paradigm for homological dynamics, whereby cellular spaces are represented up to weak homotopy equivalence by their $T_0$ quotients (obtained by contracting each cell to a single point), and dynamics over the former may be represented by multivalued maps or multi-vector fields over the latter, the weakened topology offers a compatible way of expanding these insights (e.g., see Section~\ref{hybrid Conley}) to general hybrid state spaces. 
%
%ALL: DO YOU THINK THIS PARAGRAPH OUGHT TO BE SPLIT INTO OUTCOMES AND FUTURE DIRECTIONS, SEEING AS THE OUTCOMES PART IS ONLY JUST NOW BEING WRITTEN UP? Thanks, --Dan.

\vspace*{0.3cm}

\subsection{Data-driven learning for open hybrid system (Bloch, Ghaffari)}
We are developing methods for learning hybrid systems. This builds on our 
earlier work of \cite{teng2024generalized} on using bilevel optimization 
methods for analyzing generalized open systems. This will also build on
recent work for learning e.g. Hamiltonian and Lie-Poisson systems. This is a
complex problem as it involves learning the Hamiltonian or Lagrangian structure,
the dissipative (e.g. metric) structure and the switching regimes. We have 
preliminary results on the linear case of switching. The goal is learn the different structures succesively. This work is also related to our general work on smooth and discrete metriplectic systems that we developed in \cite{bloch2024metriplectic}.

In the work of \cite{teng2024generalized}, we extend the classical Hamiltonian framework by creating a generalized metriplectic system that captures nonconservative dissipation. We achieve this by introducing a concept of free energy, akin to what we see in thermodynamic systems, through the relaxation of traditional constraints. To tackle the challenge of system identification, we employ a bilevel convex optimization method, allowing us to accurately identify the underlying metric and entropy from system observations. Our approach is validated through numerical examples, where we demonstrate its effectiveness in both 2D Hamiltonian systems and systems on SO(3). 

\subsection{HJB + Zeno (Vasudevan, Clark)}

\paragraph{Outcomes} Classical tools for hybrid optimal control, such as the hybrid maximum principle and dynamic programming formulations via the Hamilton-Jacobi-Bellman (HJB) equation, typically assume that discrete events are uniformly separated. 
In particular, the hybrid maximum principle fails in the presence of Zeno trajectories, as shown in prior work.
In this project, we explore the applicability of the HJB framework to hybrid systems that exhibit Zeno behavior.
We consider the controlled bouncing ball as a canonical testbed: a system whose trajectories necessarily undergo infinitely many impacts as they converge to a rest configuration. 
By formulating the hybrid HJB equation with appropriate boundary conditions at the guard and its image, we are able to define a value function that remains meaningful even as trajectories approach the Zeno limit. 
Using a semi-Lagrangian upwind numerical scheme, we compute the value function and investigate how optimal controls behave in the Zeno regime. 
Our results indicate that it is possible to approximate value functions for such systems despite the breakdown of traditional necessary conditions, opening a new pathway for optimal control of Zeno systems.


\paragraph*{Future Directions} We are currently validating our numerical algorithm on both simulated and physical systems to confirm its correctness in approximating the hybrid HJB value function near Zeno states. 
One of our goals is to characterize the qualitative structure of the value function near the Zeno point and to identify features such as sensitivity to initial conditions and the emergence of multiple local minima corresponding to different impact patterns.
We are also exploring how this approach generalizes to other hybrid systems, including those with nonholonomic constraints or impacts between multiple bodies.

\subsection{Mechanics and control with collision (Bloch, Clark)}
This work builds on our previous work on the dynamics  of mechanical 
systems with collisions. We consider both Hamiltonian and nonholonomic 
systems and collisions with various barriers as well as collisions 
between objects. In past work (\cite{clark2019bouncing}) we considered the dynamics of a rolling disk 
colliding with a convex barrier such a circle or ellipse generalizing 
standard work on billiard collisions. Also, these types of systems arise from Pontryagin's maximum principle for a general class of hybrid dynamical systems (\cite{clark_oc}). We are generalizing this to 
the dynamics of other nonholonomic systems such as the Chaplygin sleigh 
and analyzing integrability. In addition we are analyzing the dynamics 
of systems under gravitational forces and examining  elastic and 
inelastic collisions as well as the stability of the resulting contact 
equilibria.  This will generalize classical results on the stability of 
contact equilibria to the dynamics setting. 

\subsection{Hybrid holonomy (Guralnik, Clark)}
%
A system with intermittent contact/impacts can achieve richer motion than its constituent components can, e.g. bipedal locomotion can be achieved as a sequence of inverted pendulums while the later is incapable of locomotion. This motivates the concept of a \textit{hybrid principal bundle}.

A principal bundle is a space which consists of internal (shape) and external (structure group) variables. The shape and group variables are coupled through a choice of a connection on the bundle. This offers a convenient framework for studying locomotion as the set of possible net changes in the external variables are achievable from periodic motions in the shape variables (holonomy group) by imposing the constraints from the specified connection. When no net change in the group variables is achievable (the holonomy group is trivial), the connection is called \textit{flat}.

As seen with legged locomotion, non-smooth concatenation of flat connections need not remain flat (as walking comprised of multiple steps). This observation may be case in the framework of hybrid principal bundles, where we have shown that the holonomy group is well-defined provided the guards are defined by constraints on the shape space. Moreover, piece-wise flat trajectories can be well-approximated by curved ones, e.g. walking by rolling and a peg-in-a-slot by the Chaplygin sleigh \cite{hybrid_holonomy}. More formally, we aim to express this observation in terms of metric limits, where the curved principal bundle is the limit of hybrid principal bundles (with flat components). A major challenge is to determine a systemic way to decompose an arbitrary shape space such that the metric limit may be applied. A broad formulation of the notion of a hybrid bundle is required that would enable the use of arbitrarily fine open covers of the shape space and the study of the associated limits. Such a formulation would be an extension of the topological hybrid framework being developed under Section~\ref{sec:homologicaldynamicsHS}.

% \dg{A system with intermittent contact/impacts can achieve richer motion than its constituent components can, e.g. bipedal locomotion can be achieved as a sequence of inverted pendulums while the latter is incapable of locomotion.
% This motivates the concept of a \textit{hybrid principal bundle}.}

% \dg{A principal bundle is a space which consists of internal (shape) and external (structure group) variables. The shape and group variables are coupled through a choice of a connection on the bundle.
% This offers a convenient framework for studying locomotion as the set of possible net changes in group variables achievable from periodic motions of the shape variables (holonomy group), subject to the constraints imposed by the specified connection.
% When no net change in the group variables is achievable through shape changes (the holonomy group is trivial), the connection is referred to as \textit{flat}.}

% \dg{
% As seen with legged locomotion, non-smooth concatenation of flat connections need not remain flat.
% This observation may be cast in the framework of hybrid principal bundles, where we have shown that the holonomy group is well-defined provided the guards are defined by constraints set on the shape space.
% Moreover, a hybrid principal bundle with flat components can be well-approximated by a curved bundle, e.g. walking by rolling and a peg-in-a-slot by the Chaplygin sleigh.
% (WILL, IS THERE ANYTHING MORE WE WOULD LIKE TO SAY AS FAR AS ACHIEVEMENTS GO? \cite{hybrid_holonomy})
% More formally, we aim to express this observation in terms of metric limits, where the curved principal bundle is the limit of hybrid principal bundles with flat components, as the flat components are compressed indefinitely in the sense of the diameters of their bases being shrunk to zero.
% A major challenge is that is may be impossible to decompose an arbitrary shape space into arbitrarily small polyhedra (the polyhedral structure corresponding to impacts).
% Instead, a broader formulation of the notion of hybrid bundle is required that would enable the use of arbitrarily fine open covers of the shape space and the study of the associated limits.
% Such a formulation would also need to be consistent with the more general hybrid systems framework being developed under Section~\ref{sec:homologicaldynamicsHS}.}
\vspace*{0.3cm}

\section{Openness and Composition}
\paragraph*{Outcome}
\subsection{Hybrid dynamic networks (Bloch)}

This work investigates the stability and stabilization of diffusively coupled network dynamical systems. We leverage Lyapunov methods to analyze the role of coupling in stabilizing or destabilizing network systems. We derive critical coupling parameter values for stability and provide sufficient conditions for asymptotic stability under arbitrary switching scenarios, thus highlighting the impact of both coupling strength and network topology on the stability analysis of such systems. Our theoretical results are supported by numerical simulations. Some of this work may be found in \cite{mouyebe2025coupling}.  We are currently 
extending these ideas to hypergraph networks both with linear and nonlinear coupling, see e.g \cite{pickard2024kronecker}. We are also applying these ideas 
to large data sets and to hybrid vector fields analyzed in the DSGRN framekwork as discussed below. 
\subsection{Stochastic hybrid system (Lee, Clark)}
As hybrid systems are comprised of multiple interlocking components, stochasticity may be interpreted in many different ways. In \cite{stochastic-hybrid}, two stochastic cases are presented extending the deterministic results of \cite{OpSh_fp} to systems whose continuous dynamics are stochastic and discrete dynamics are either deterministic or stochastic. The Koopman operator encodes the expected value of an observable while the Frobenius-Perron operator is its adjoint and evolves densities. The generator for these operators is comprised as a partial differential equation (PDE) (arising from the continuous dynamics) and boundary conditions (arising from the discrete dynamics). The induced boundary conditions are qualitatively different depending on whether the jumps are stochastic or deterministic. Numerical experiments utilizing Monte Carlo simulation are performed to validate the analytic predictions from the hybrid PDE. We are extending these restults to higher dimensions and to general manifolds.

\subsection{Geometric numerical integration of stochastic hybrid system (Lee)}
\paragraph*{Future Directions}

\subsection{Geometric modeling of deterministic state uncertainty (Guralnik, Lee)}
%
\dg{In a variety of control applications, the absence of state feedback necessitates the development of dynamic state estimates, giving rise to a \emph{region of uncertainty}: a time-varying (usually growing) ball, representing an upper estimate of all the states attainable from the set of initial conditions via an admissible control.
%
Planning and control methods relying on such estimates are inherently conservative, representing minimal knowledge and observability of system behavior.
%
At the same time, they are unavoidable in risk-averse scenarios where probabilistic guarantees of mission success are unacceptable due to high failure costs.
%
}

\dg{
In practice, intermittent measurements of the state may be executed, providing discrete opportunities to ``chip away'' at the region of uncertainty, resulting in an improved, yet geometrically more complex, dynamic estimate of the state evolving under hybrid, shape-dependent dynamics.
%
It turns out that the existing control methods may be expanded to take into account uncertainty regions of time-varying geometry.
%
Current work focuses on developing a theoretical understanding of possible performance guarantees in the ideal case, when the continuous evolution of the uncertainty region is known in advance, as well as of a more computationally plausible method where the uncertainty region is estimated from above by a cover consisting of $N$ balls.
%
Extensions to other efficient shape representations, such as polynomial zonotopes, will be considered as well.
%
We aim to study the tradeoffs between the complexity of the representation (e.g., $N$, in the case of ball covers) and the quality of the resulting hybrid controllers, taking into account possible topological transitions: for example, when the uncertainty region fractures (becomes disconnected), a state update could be triggered in an attempt to reduce the region to a single component.
%
}
\vspace*{0.3cm}

\subsection{Hybrid dynamic network for DSGRN (Bloch, Mischaikow, Kalies, Guralnik)}\label{dynamic networks in DSGRN}
%
\cite{mouyebe2025coupling} investigated the stability and stabilization of coupled network dynamics using Lyapunov methods to prescribe the consequences of coupling effects.
%
An interesting challenge addressed are switched network systems, where the network structure can change over time.
%
We will study coupled network dynamics from a homological perspective.
%
We plan to extend the theoretical framework in DSGRN (\cite{gameiro2024globaldynamicsordinarydifferential}) to recast the problem of changes in network topology as changes in parameters of a larger network.
%
This allows one to characterize the global dynamics of switching networks by constructing combinatorial models that preserve algebraic topological invariants across different network configurations.
%
This approach will enable us to systematically analyze how network reconfiguration affects equilibria, periodic orbits, and connecting orbits, while identifying critical parameter regions where qualitative changes in system behavior occur.
%
%\dg{ALL, I am happy with this version as is, but I must still ask: would you like me to throw strategy spaces into the mix here? Thanks, --Dan. Tony: I would be okay if you added that. }
\vspace*{0.3cm}


\section{Scalability and Verification}
\paragraph*{Outcome}

\subsection{Hybrid optimal control with affine geometric heat flow (Vasudevan)}

\paragraph*{Outcomes} Recent work by the Co-PIs has introduced a new PDE-based framework for trajectory optimization called the affine geometric heat flow (AGHF). 
The central idea is to evolve a trajectory toward optimality by treating it as the solution to a parabolic partial differential equation on a 2D domain, where the pseudo-time variable drives the deformation of an initial path into one that satisfies both the system dynamics and boundary conditions. 
One can prove that under mild conditions on the dynamics, that this approach will converge to a feasible trajectory if one exists.
This approach offers a scalable alternative to traditional Hamilton-Jacobi or Pontryagin-based methods, which can be computationally intractable for high-dimensional systems. 
Notably, the AGHF does not require solving over the full state space, and instead operates on the trajectory as a whole.

Building on this theoretical foundation, the PHLAME framework \cite{adu2025phasing} introduced a pseudospectral implementation using Chebyshev collocation methods and spatial algebra to accelerate the solution process, achieving order-of-magnitude improvements in runtime. 
BLAZE \cite{adu2025bring} extends this further by generalizing the objective function to allow arbitrary cost functionals and incorporating a two-phase optimization procedure. This two-phase method first ensures feasibility—even from constraint-violating initial guesses—and then refines the trajectory for optimality. 
These methods have been demonstrated on a variety of systems, including 20+ DoF manipulators and humanoid robots, achieving constraint-satisfying, dynamically feasible trajectories in as little as 2–5 seconds. 
Importantly, input and state constraints, including actuator limits and collision avoidance, are natively handled in the optimization.
We have shown how to extend these approaches to hybrid systems with impacts when one assumes that the number and sequence of contacts is specified a priori. 

\paragraph*{Future Directions} Our ongoing research seeks to extend the AGHF framework to hybrid systems with impacts and discrete transitions where the contact sequence is not specified a priori.
 One promising direction is to couple the AGHF evolution with complementarity style formulation, incorporating reset maps directly into the PDE or into the phase structure used in BLAZE.
Additionally, we are investigating whether the two-phase approach developed in BLAZE—first projecting to the feasible set, then optimizing—can be extended to systems with Zeno behavior or where the number and timing of impacts is not known in advance. 
This would allow us to treat problems where the trajectory involves a variable number of discrete events, such as multi-bounce systems or hybrid transitions under uncertainty. 
Our goal is to create a numerically robust framework for hybrid optimal control that combines the computational advantages of AGHF with the structural insights of hybrid systems theory.

\subsection{Hybrid mean field game (Lee)}
\subsection{DHAL: Discrete hybrid automata learning (Ghaffari, Mischaikow, Kalies, Clark)}

We developed the Discrete-time Hybrid Automata Learning (DHAL)~\cite{liu2025discrete} framework to address the complexities of open hybrid dynamical systems, specifically targeting robotic skateboarding as an application. This approach integrates hybrid automata models with reinforcement learning, enabling robots to adapt to dynamic environments, such as snowy slopes, without predefined trajectories or event labels. It enables explainable interaction with the surroundings in contact-rich scenarios via an unsupervised mode identification.

We validated the DHAL framework using a quadrupedal robot capable of executing complex skateboarding maneuvers. Our multi-critic training strategy, which includes rewards for gliding, pushing, and transitioning from simulations to real-world conditions, demonstrates the framework's scalability. The robot effectively navigated diverse terrains and disturbances, maintaining its performance under challenging conditions. This work advances the scalability and verification of open hybrid systems by explicitly identifying task-level motion modes.

In the future, we are interested in extending the latent space homological analysis to this hybrid setting. 

\paragraph*{Future Directions}
%\subsection{? Hybrid, non-holonomic optimal control (Bloch)}
\subsection{Hybrid rigid body optimal control (Ghaffari, Vasudevan, Lee)}
We developed certifiable globally optimal trajectory optimization for multi-body systems~\cite{teng2024convex} and a fast Riemannian solver on Lie groups for direct trajectory optimization~\cite{teng2025riemannian}. The two works are complementary, i.e., global and local search. We will work on extending these to a hybrid setting, e.g., mechanical systems with contact.

\subsection{Generalized DSGRN with control (Guralnik, Mischaikow, Kalies)}
%
The power of DSGRN is that it starts with a network model that indicates which variables are influencing which other variables.
%
From the network a cubical complex and a parameter space (represented by a graph) is constructed.
%
This purely combinatorial/homological structure is used to perform computations to characterize the dynamics over different parameter regions.
%
The combinatorial model for dynamics is constructed to model a flow generated by an ODE.
%
After the computations are performed, this abstract structure is given a geometric realization in such a way that one can draw rigorous conclusions about the continuous process.
%

Our project is to build a similar framework for hybrid systems.
%
We have a mathematical structure to work with: the graph in the product complex of the combinatorial model.
%
For the continuous part of the dynamics we expect it will take a similar form as DSGRN and 
%
the model of behavior at guards should follow that of combinatorial models for dynamics generated by continuous maps.
%
One challenge is that our algorithms are not designed work with this type of hybrid system and preliminary work suggests that the homological interpretation of dynamics can be significantly different from that of traditional dynamical systems.
%

Another challenge is that discretization limits the possibilities for control, as input values are selected as functions of just the discrete state.
%
In the case of DSGRN, regarding some of the system parameters as control inputs gives rise to a discrete transition system embedded in the product of the cubical complex with the parameter space.
%
Continuity of control may then be approximated using the graph structure of parameter space, but this kind of control---whether implemented in reality using piecewise-constant inputs or otherwise---is inherently hybrid.
%
Consequently, two goals arise: obtaining a valid extension of DSGRN's way of discretizing parametrized dynamics over general open hybrid systems, and obtaining means for analysis of closed-loop behaviors in such systems.
%
In particular, one direction of current inquiry is the understanding of controllability properties of DSGRN instances as a function of the underlying interaction/interconnection graph, using strategy spaces, and in line with the goals of Section~\ref{dynamic networks in DSGRN}.
%

\vspace*{0.3cm}



\subsection{Invariant filtering for hybrid dynamic network (Lee, Ghaffari)}
We will study the extension of invariant and equivariant observers to hybrid systems, e.g., under a change of locomotion mode or moving ground. A more general setting would be hybrid systems in non-inertial frames, and we may explore it. We developed a maximum entropy filter~\cite{teng2025max} that can track polynomial systems under arbitrary noise distributions. We plan to extend the framework to stochastic hybrid systems and their connection with Lie theory for equivariant systems. 

\end{document}

