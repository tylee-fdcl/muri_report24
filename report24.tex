\documentclass[letterpaper,11pt]{article}

\usepackage[margin=1in]{geometry}
\usepackage{times}
\usepackage[compact]{titlesec}
\usepackage{amsmath,amssymb,amsthm}
\usepackage{graphicx}
\usepackage{subcaption}
\usepackage{hyperref}
\usepackage[T1]{fontenc}
\usepackage[giveninits=true]{biblatex}
%\usepackage[style=mla]{biblatex}
\usepackage{tabularx}
\usepackage{cleveref}
\usepackage{tikz}

\usetikzlibrary{arrows.meta}
\usetikzlibrary{shapes.geometric}
\usetikzlibrary{shapes.multipart}
\usetikzlibrary{cd}

\addbibresource{ref23.bib}

\graphicspath{
    {Figs/},{figures/}
}

\titleformat*{\paragraph}{\itshape}

\newcommand{\cF}{{\mathcal{F}}}
\newcommand{\cX}{{\mathcal{X}}}

\begin{document}

\thispagestyle{empty}
\noindent\textbf{ANNUAL REPORT}\\[1cm]
\centerline{\textbf{\Large Unified Large-Scale Theoretical and Computational Frameworks}}
\centerline{\textbf{\Large for Invariance and Composition of Open Hybrid Dynamical Systems}}\\[1cm]

\renewcommand\arraystretch{1.5}
\begin{tabularx}{1.0\textwidth}{>{\bfseries}lX}
Award Number & FA9550-23-1-0400\\
Report Type & Annual\\
Reporting Period & July 2, 2024 - July 1, 2025\\
Distribution Statement & Distribution A - Approved for public release\\
Program Officer Name & Dr. Frederick Leve\\
Principal Investigator Name & Dr. Taeyoung Lee\\
Project Title & Unified Large-Scale Theoretical and Computational Frameworks for Invariance and Composition of Open Hybrid Dynamical Systems\\
%
Abstract & TBA
%This project is to establish both theoretical and computational frameworks to analyze and certify the intriguing behaviors of complicated open hybrid dynamical systems. 
%In particular, the objective is to identify and construct the inherent structures of hybrid dynamics, such as topological properties and invariances, that can be preserved under the interaction with uncertain environments and composition over a complex network. 
%During YR1, we investigated 15 topics in the area of geometry, topology, openness, scalability, and category of hybrid systems. 
\end{tabularx}

\clearpage\newpage

\tableofcontents

\clearpage\newpage
\setcounter{page}{1}
\section{Introduction}
This project will develop theoretical and computational frameworks to analyze and certify the behavior of complex open hybrid dynamical systems, such as embodied artificial intelligence,  quantum systems, and biological systems. 
The objective is to identify and construct the inherent structures of hybrid dynamics, such as topological properties and invariances, that can be preserved under the interaction with an uncertain environment and composition over a complex network.
The novelty lies in establishing a trustworthy computational foundation that is carefully constructed in conjunction with the underlying geometry, leading to a significant generalization capacity and computational efficiency in understanding the global topological properties of complex, composable hybrid systems. 

This will be achieved by multidisciplinary collaborative efforts in dynamical systems and theoretical computer science. 
In particular, we focus on three research thrusts: discovery of geometric and topological structures; investigation of the uncovered structures in open, uncertain environments and composition; extension to gene regulatory networks/multi-agent systems and further generalization to scalable composition and certification, as illustrated in \Cref{fig:overview}.

\usetikzlibrary{matrix, positioning, fit, backgrounds}
\begin{figure}[h]
    \scriptsize
    \centerline{
        \begin{tikzpicture}[
            mymatrix/.style={matrix of nodes, nodes=typetag, row sep=7pt, align = center},
            typetag/.style={draw, thick, fill=white, inner sep=1.5ex, anchor=west, text width = 0.27\textwidth, align = center},
            mycontainer/.style={draw=gray, inner sep=1ex},
            title/.style={draw=none, fill=none, inner sep=0pt, align=center, font=\bfseries},
            analytic/.style={draw},
            mixed/.style={draw, dash dot},
            comp/.style={draw, dotted}
            ]
            \matrix[mymatrix] (GT) {
                |[title]| (i) Geometry \& Topology\\
                |[analytic]| {Geometry and Topology\\ (Bloch, Clark)} \\
                |[mixed]| {Homological Dynamics\\ (Mischaikow, Kalies)} \\
                |[comp]| { Topological Dynamics Learning\\ (Ghaffari, Bloch, Clark) }\\ 
            };
            \matrix[mymatrix, right=20pt of GT.north east, matrix anchor=north west] (OS) {
                |[title]| (ii) Openness \& Scalability\\
                |[analytic]| { Geometry and Topology\\ (Bloch, Clark)} \\
                |[mixed]| { Stochastic Hybrid Network\\ (Lee, Mischaikow, Kalies)} \\
                |[analytic]| { Unified Composition Framework \\ (Guralnik, Vasudevan, Ghaffari) }\\ 
            };
            \matrix[mymatrix, right=20pt of OS.north east, matrix anchor=north west, row sep=5pt] (EV) {
                |[title]| (iii) Extension \& Verification\\
                |[mixed, inner sep=3pt]| { Gene Regulatory Networks\\[-0.05cm] (Mischaikow, Kalies)} \\
                |[mixed, inner sep=3pt]| { Cooperative Multi-agent Systems\\[-0.05cm] (Lee, Guralnik)} \\
                |[comp, inner sep=3pt]|{ Scalable Composition\\[-0.05cm] (Vasudevan, Ghaffari, Guralnik)} \\
                |[comp, inner sep=3pt]| { Formal Verification\\[-0.05cm] (Vasudevan, Lee, Mischaikow) }\\ 
            };
            \node[mycontainer, fit=(GT)] (GTBox) {};%, label={[text width=0.27\textwidth] below:discover topological properties with three distinct approraches}] (GTBox) {};
            \node[mycontainer, fit=(OS)] (OSBox) {};
            \node[mycontainer, fit=(EV)] (EVBox){};

            \begin{scope}[on background layer]
                \node[mycontainer, fit=(GTBox), fill=gray!10, inner sep = 0pt] {};
                \node[mycontainer, fit=(OSBox), fill=gray!10, inner sep = 0pt] {};
                \node[mycontainer, fit=(EVBox), fill=gray!10, inner sep = 0pt] {};
            \end{scope}

            \draw[arrows={-Triangle[angle=30:6pt]}, thick] (GTBox) -- (OSBox);
            \draw[arrows={-Triangle[angle=30:6pt]}, thick] (OSBox) -- (EVBox);
            \draw[arrows={-Triangle[angle=30:6pt]}, thick] (EVBox.south) -- ++(0,-8pt) -- ([yshift=-8pt]GTBox.south) -- (GTBox.south);
        \end{tikzpicture}
    }
    \DeclareRobustCommand\captionsolid{\tikz[baseline=-0.6ex]\draw[thick] (0,0)--(0.5,0);}
    \DeclareRobustCommand\captiondashdot{\tikz[baseline=-0.6ex]\draw[thick, dash dot] (0,0)--(0.5,0);}
    \DeclareRobustCommand\captiondotted{\tikz[baseline=-0.6ex]\draw[thick, dotted] (0,0)--(0.5,0);}
    \caption{Proposed research: (i) we will discover geometric and topological properties of hybrid systems with three distinct approaches: theoretical analysis (\captionsolid), mixed theoretical analysis and computation (\captiondashdot), and computation (\captiondotted); (ii) next, we will investigate how such topological properties are affected by environments, uncertainties, and large-scale compositions; (iii) these will be further extended to gene regulatory networks and multi-agent systems, and they will be verified with formal methods.}\label{fig:overview}
\end{figure}

\newpage

\section{Geometry and Topology}
\paragraph*{Outcome}
\subsection{Linear affine hybrid system (Clark)}
\subsection{Hybrid Conley index theory (Kalies)}
\subsection{Data-driven learning for hybrid systems (Mischaikow, Clark)}
\subsection{Homological dynamics of hybrid system (Mischaikow, Guralnik)}
\paragraph*{Future Directions}
\subsection{Data-driven learning for open hybrid system (Bloch, Ghaffari)}
\subsection{HJB + Zeno (Vasudevan, Clark …)}
\subsection{Mechanics and control with collision (Bloch, Clark)}
\subsection{Hybrid holonomy (Guralnik, Clark)}
\vspace*{0.3cm}

\section{Openness and Composition}
\paragraph*{Outcome}
\subsection{Hybrid dynamic networks (Bloch)}
\subsection{Stochastic hybrid system (Lee, Clark)}
\subsection{Geometric numerical integration of stochastic hybrid system (Lee)}
\paragraph*{Future Directions}
\subsection{Geometric modeling of deterministic state uncertainty (Guralnik, Lee)}
\subsection{Hybrid dynamic network for DSGRN (Bloch, Mischaikow, Kalies, Guralnik)}
\vspace*{0.3cm}


\section{Scalability and Verification}
\paragraph*{Outcome}
\subsection{Hybrid optimal control with affine geometric heat flow (Vasudevan)}
\subsection{Hybrid mean field game (Lee)}
\subsection{DHAL: Discrete hybrid automata learning (Ghaffari, Mischaikow, Kalies, Clark)}
\paragraph*{Future Directions}
%\subsection{? Hybrid, non-holonomic optimal control (Bloch)}
\subsection{Hybrid rigid body optimal control (Ghaffari, Vasudevan, Lee)}
\subsection{Generalized DSGRN with control (Guralnik, Mischaikow, Kalies)}
\subsection{Invariant filtering for hybrid dynamic network (Lee, Ghaffari)}


\end{document}

